\documentclass[a4paper,10pt,french]{article}
\newcounter{question}%[section]
\newcounter{subquestion}%[section]
\newcommand{\question}{%
  \setcounter{subquestion}{0}
  \refstepcounter{question}%
  \paragraph{Question \thequestion}%
}

% Préambule; packages qui peuvent être utiles
   \RequirePackage[T1]{fontenc}        % 
   \RequirePackage{babel,indentfirst}  % Pour les césures correctes,
                                       % et pour indenter au début de chaque paragraphe
   \RequirePackage[utf8]{inputenc}   % Pour pouvoir utiliser directement les accents
                                     % et autres caractères français
   \RequirePackage{lmodern,tgpagella} % Police de caractères
   \textwidth 17cm \textheight 25cm \oddsidemargin -0.24cm % Définition taille de la page
   \evensidemargin -1.24cm \topskip 0cm \headheight -1.5cm % Définition des marges
   \RequirePackage{latexsym}                  % Symboles
   \RequirePackage{amsmath}                   % Symboles mathématiques
   \RequirePackage{tikz}   % Pour faire des schémas
   \RequirePackage{graphicx} % Pour inclure des images
   \RequirePackage{listings} % pour mettre des listings
% Fin Préambule; package qui peuvent être utiles

\title{Rapport de TP 4MMAOD : Justification optimale de texte}
\author{
NOM Prénom étudiant$_1$ (groupe étudiant$_1$) 
\\ NOM Prénom étudiant$_2$ (groupe étudiant$_2$) 
}

\begin{document}

\maketitle

%%%%%%%%%%%%%%%%%%%%%%%%%%%%%%%%%%%%%%%%%%%%%%
\paragraph{\em Préambule}
{\em Ce patron de rapport est donné à titre d'exemple pour répondre aux questions 4, 5 et 6 de l'énoncé.
Vous pouvez soit compléter ce patron (complètement ou partiellement selon les résultats fournis par votre programme)
soit rédiger votre propre rapport (qui ne doit pas excéder 2 à 3 pages). Dans tous les cas l'évaluation suit
le barème indiqué ci-dessous.
\begin{itemize} 
   \item Compléter ce patron de rapport en supprimant toutes les phrases en italique qui ne doivent pas apparaître dans le rapport pdf.
   \item Le rapport doit tenir sur 2 à 3 pages au maximum.
   \item Barème (sur 20 points) : Programme=4 pts ; Tests=10 points; Rapport= 6 points = 5 points + {\bf 1 point} pour la qualité globale du rapport\,: présentation, concision et clarté de l'argumentation.
   \item Détail du barème du programme sur 4 points :  lisibilité du code=1 point; commentaires/doxygen=1 point; 
      efficacité de la programmation=1 point; gestion de la mémoire=1 point (mmap, allocation et libération)
\end{itemize}
}

%%%%%%%%%%%%%%%%%%%%%%%%%%%%%%%%%%%%%%%%%%%%%%
\section{Principe de notre  programme (1 point)}
{\em Mettre ici une explication brève du principe de votre programme en  précisant la méthode implantée (récursive, itérative) et les
choix d'implanattaion effectués (valeurs mémorisées, lecture du fichier (mmap?), écriture, etc).
\\
} 

%%%%%%%%%%%%%%%%%%%%%%%%%%%%%%%%%%%%%%%%%%%%%%
\section{Analyse du coût théorique (1.5 points)}
{\em Donner ici l'analyse du coût théorique de votre programme en fonction du nombre $n$ de caractères, du nombre $m$ de mots et de
la longueur $L$ d'une ligne.
 Pour chaque coût, donner la formule qui le caractérise (en justifiant brièvement pourquoi cette formule correspond à votre programme), 
 puis l'ordre du coût en notation $\Theta$ de préférence, sinon $O$.}

\noindent{NB: Le travail (i.e. nombre d'opérations) est mesuré en nombre  de comparaisons des coûts de justification (dans le calcul du Min de l'équation de Bellman).}

  \subsection{Nombre  d'opérations (comparaions/min)  en pire cas\,: }
    \paragraph{Justification\,: }
%    {\em La justification peut être par exemple de la forme: \\ 
%       "Le programme itératif contient les boucles $k_1=...$, $k_2= ...$ etc correspondant à la somme 
%      $$C(n_1, n_2, c_1, c_2) = \sum_{k_1=...}^{...} ... \sum ... + \sum_{i=...}^{...} ...$$ 
%      somme que nous avons calculée (ou majorée) par la technique de  ... " \\
%      ou  encore\,:  \\
%      "les appels récursifs du programme permettent de modéliser son coût par le système d'équations aux récurrences 
%      $$C(k_1, k_2) = ...  \mbox{~avec~les~conditions~initiales~....~} $$
%      Le coût indiqué est obtenu en résolvant ce système par la méthode de  .... "
%    } 
  \subsection{Place mémoire requise\,: }
    \paragraph{Justification\,: }

  \subsection{Nombre de défauts de cache sur le modèle CO\,:}
    \paragraph{Justification\,: }


%%%%%%%%%%%%%%%%%%%%%%%%%%%%%%%%%%%%%%%%%%%%%%
\section{Compte rendu d'expérimentation (2.5 points)}
  \subsection{Description de la machine et conditions expérimentalesi (0.5 point) }
     {\em  Décrire les conditions permettant la reproductibilité des mesures. 
      \\ Décrire la machine utilisée pour les tests (par exemple {\tt pcserveur.ensimag.fr} ou une autre machine auquel cas indiquer processeur et sa fréquence, la mémoire, le système d'exploitation). \\
      Préciser si la machine était monopolisée pour un test ou si 
       d'autres processus ou utilisateurs étaient en cours d'exécution.\\
       Préciser comment les mesures de temps ont été effectuées (fonction appelée, par exemple {\tt getrusage} ou la commande unix {\tt time}) et l'unité de temps; en particulier, 
       préciser comment les 5 exécutions pour chaque test ont été faites (par exemple si le même test est fait 5 fois de suite, ou si les tests sont alternés entre
       les mesures, ou exécutés en concurrence etc). 
       
     }

  \subsection{Mesures expérimentales (1 point)}
\noindent    {\em Compléter le tableau suivant par les temps d'exécution 
              (temps minimum, maximum et moyen sur 5 exécutions)
mesurés pour les longueurs de ligne ci-dessous sur le benchmark : \\
{\tt /matieres/4MMAOD6/Benchmark/ALaRechercheDuTempsPerdu-1paragraphe-ISO-8859-1.in } \\
Seules les mesures pour lesquelles le programme marche doivent être indiquées.
Si le benchmark ne passe pas, vous pouvez en prendre ou en construire  un autre sur lequel votre programme fonctionne
et avec des valeurs de $M$ mettant en évidence des défauts de cache éventuels.
    }

    \begin{figure}[h]
      \begin{center}
        \begin{tabular}{|l||r||r|r|r||r||r||}
          \hline
          \hline
            longueur    & valeur    de       & temps     & temps   & temps  & temps & temps \\
            ligne ($M$) & justification      & elapsed   & elapsed & elapsed& user  & system \\
                        &                    & min       & max     & moyen  & moyen & moyen \\
          \hline
          \hline
            200  &      &     &     &   &  &  \\
          \hline
            400 &      &     &     &   &  &  \\
          \hline
            600 &      &     &     &   &  &  \\
          \hline
            800 &      &     &     &   &  &  \\
          \hline
            1000 &      &     &     &  &  &   \\
          \hline
            1200 &      &     &     &  &  &   \\
          \hline
            1400 &      &     &     &  &  &   \\
          \hline
            1600 &      &     &     &  &  &   \\
          \hline
            1800 &      &     &     &  &  &   \\
          \hline
            2000 &      &     &     &  &  &   \\
          \hline
          \hline
        \end{tabular}
        \caption{Mesures des temps minimum, maximum et moyen de 5 exécutions.
{\em Seules sont indiquées les valeurs de $M$ significatives et pour lesquelles le programme s'exécute sans erreur et en temps raisonnable.} }
        \label{table-temps}
      \end{center}
    \end{figure}

\noindent    {\em Compléter le tableau suivant par les mesures  d'exécution  données 
pour les longueurs $M$ de ligne ci-dessous 
par la commande {\em valgrind} : \\
{\tt 
valgrind  --tool=cachegrind   ./AODjustify  $<M>$  ALaRechercheDuTempsPerdu-1paragraphe-ISO-8859-1 }
    }

    \begin{figure}[h]
      \begin{center}
        \begin{tabular}{|l||r||r|r|r||}
          \hline
          \hline
            longueur    & \#instructions    & \#défauts  & \#défauts   & \#défauts \\
            ligne ($M$) & (travail)         & au total   & en lecture  & en écriture \\
          \hline
          \hline
            200  &      &     &     &     \\
          \hline
            400 &      &     &     &     \\
          \hline
            600 &      &     &     &     \\
          \hline
            800 &      &     &     &     \\
          \hline
            1000 &      &     &     &     \\
          \hline
            1200 &      &     &     &     \\
          \hline
            1400 &      &     &     &     \\
          \hline
            1600 &      &     &     &     \\
          \hline
            1800 &      &     &     &     \\
          \hline
            2000 &      &     &     &     \\
          \hline
          \hline
        \end{tabular}
        \caption{Mesures des défauts de cache avec {\tt valgrind  --tool=cachegrind}.
{\em Seules sont indiquées les valeurs de $M$ significatives et pour lesquelles {\tt valgrind} s'exécute sans erreur et en temps raisonnable.} }
        \label{table-cache}
      \end{center}
    \end{figure}
\subsection{Analyse des résultats expérimentaux (1 point)}
{\em Donner  une réponse justifiée  à la question\,: 
les  mesures expérimentales correspondent-elles  à l'analyse théorique (nombre d’opérations et défauts de cache) ?
}


\end{document}
%% Fin mise au format
